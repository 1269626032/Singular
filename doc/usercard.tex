% $Id: usercard.tex,v 1.7 2005-09-29 13:10:32 Singular Exp $

%
% usercard.tex - Singular user quick reference card.
%

\input singcard.tex

\centerline{\hbf SINGULAR Quick Reference}

\centerline{\srm {\ssc Singular} Version 3-0}

\bigskip

Do not forget to terminate all commands with a {\tt ;} (semicolon)!

In particular if \Singular\ prints the continuation prompt {\tt .}
(peri\-od) instead of the regular command prompt {\tt >}, then it
waits for a command to be terminated by a {\tt ;}.  If that does
not help, try one or more {\tt "} or {\tt \char`}} to close an
opened string or block.

Comments start with {\tt //} and extend to end of line.

\smallskip
Some of the topics concerning interactive use are system dependent.

\sec Starting SINGULAR(2.5cm)
Singular& 			start \Singular\cr
Singular {\it file} \rep&	read {\it files\/} and prompt for further commands\cr
Singular --help&		print help on command line options and exit\cr
\endsec

\sec Stopping SINGULAR()
quit;&				exit \Singular; also {\tt exit;} or {\tt \$}\cr
\ctl c&				interrupt \Singular\cr
\endsec

\sec Getting help()
help;&				enter online help system\cr
help {\it topic\/};&		describe {\it topic\/}; also {\tt? {\it topic\/};}\cr
\subsec{Inside the info help system:}
\ctl h&				get help on help system\cr
q&				exit from help system\cr
n\rmslash p\rmslash u&		go to next/previous/upper node\cr
m&				pick menu item by name\cr
l&				go to last visited node/exit from help on help\cr
SPC\rmslash DEL&		scroll forward/backward one page\cr
\endsec

\sec Commandline editing()
\sectext
Commandline editing is similar to that of, e.g., {\tt bash} or {\tt tcsh}:\cr
BS\rmslash\ctl d&		remove character on the left/right of cursor\cr
\ctl p\rmslash\ctl n&		get previous/next line from history\cr
\ctl b\rmslash\ctl f&		move cursor left/right\cr
\ctl a\rmslash\ctl e&		go to beginning/end of line\cr
\ctl u\rmslash\ctl k&		delete to beginning/end of line\cr
\endsec

\sec Names and objects()
\sectext
Names (= identifiers) have to be declared before they are used:\cr
\entryskip
\longentry {\it type\/} {\it name\/} \opt{= {\it expression\/}};&
				declare variable {\it name}\cr
kill({\it name\/})&		delete variable {\it name}\cr
\entryskip
\sectext
\parskip=\verysmallskipamount
Names of type {\tt number}, {\tt poly}, {\tt ideal}, {\tt vector}, {\tt
module}, {\tt matrix}, {\tt map}, and {\tt resolution} may be declared
only inside a ring.  They are local to that ring.  The same holds for a {\tt
list} if it contains an object of the above types.  All other types may be
declared at any time.  They are globally visible.

Names may consist of alphanumeric characters including {\tt \_}
(underscore) and have to start with a letter.  Capital and small letters
are distinguished.  Names may be followed by an integer expression in
parentheses, resulting in so-called {\it indexed names}.\cr
\entryskip
{\it name\/}({\it n}..{\it m\/})&
				shortcut for {\tt {\it name\/}({\it
				n\/}), $\ldots$, {\it name\/}({\it
				m\/})}\hfil\break
				(\eg {\tt ring r = 0, x(1..3), dp;})\cr
\_ {\rm (underscore)}&		refers to the value of the last expression
				printed\cr
\endsec

\vfill
\centerline{\srm \copyright 1998-2005 \qquad Permissions on back}

\eject

\sec Ring declaration()
\longentry ring {\it name\/} = {\it basefield}, ({\it ringvars\/}), {\it ordering\/};&
				\hyphenpenalty=50 \tolerance=200 declare ring
				{\it name\/} and make it the new base\-ring.
				{\it ringvars\/} has to be a list of names, the
				other items are described below.  Example:\par
				{\tt ring r = 32003, (x, y, z), dp;}\cr
\longentry qring {\it name\/} = {\it ideal\/};&
				declare quotient ring {\it name\/} of the
				current base\-ring with respect to {\it
				ideal\/}.  {\it ideal\/} has to be a standard
				basis.  Make {\it name\/} the new basering.\cr
\subsec{Examples of available {\bit basefields\/}:}
0&				the rational numbers\cr
\it p&				the finite field $Z_p$ with {\it p\/}
				elements,\par
				$2 \le p \le 2147483629$ a prime\cr
({\it p\/}\^{}{\it n}, {\it gen\/})&
				the finite field with $p^n$ elements, {\it p\/}
				a prime and\par $4 \le p^n \le 32671$. The name
				{\it gen\/} refers to some generator of the
				cyclic group of unities.\cr
({\it p}, {\it alpha\/})&	algebraic extension of $Q$ or $Z_p$ ($p =
				0$ or as above) by {\it alpha}.  The minpoly
				$\mu_{\hbox{\tit alpha}}$ for {\it alpha\/} has
				to be specified with an assignment to {\tt
				minpoly} (\eg {\tt minpoly=a\^{}2+1;}, for
				$\hbox{\it alpha}=\hbox{\tt a}$).  {\it
				alpha\/} has to be a name.\cr
({\it p}, $t_1$, $\ldots$)&	transcendental extension of $Q$ or $Z_p$
				($p = 0$ or as above) by~$t_i$.  The $t_i$ have
				to be names.\cr
real,{\it len}&			the real numbers represented by long floating 
                                point numbers of length{\it len}\cr
\endsec

\sec Term orderings()
\sectext
An {\it ordering\/} as referred to in the ring declaration may either be
a global, local, or matrix ordering or a list of these resulting in a
pro\-duct ordering.  The list may include extra weight vectors and may be
preceded or followed by a module ordering specification.\cr
\subsec{Global orderings}
lp&				lexicographical ordering\cr
dp&				degree reverse lexicographical ordering\cr
Dp&				degree lexicographical ordering\cr
wp($w_1$, $\ldots$)&		weighted reverse lexicographical ordering\cr
Wp($w_1$, $\ldots$)&		weighted lexicographical ordering\cr
&				The $w_i$ have to be positive integers.\cr
\subsec{Local orderings}
ls&				negative lexicographical ordering\cr
ds&				negative degree reverse lexicographical ordering\cr
Ds&				negative degree lexicographical ordering\cr
ws($w_1$, $\ldots$)&		general weighted reverse lexicographical ordering\cr
Ws($w_1$, $\ldots$)&		general weighted lexicographical ordering\cr
&				$w_1$ has to be a non-zero integer, every other
				$w_i$ may be any integer\cr
\subsec{Matrix orderings}
\longentry M($m_{11}$, $m_{12}$, $\ldots$, $m_{nn}$)&
				{\it m\/} has to be an invertible matrix with
				integer coeffi\-cients.  Coefficients have to be
				specified row-wise.\cr
\subsec{Product orderings}
\longentry($o_1$\opt{($k_1$)}, $o_2$\opt{($k_2$)}, $\ldots$, $o_n$\opt{($k_n$)})&
				the $o_i$ have to be any of the above orderings.
				{\tt lp}, {\tt dp}, {\tt Dp}, {\tt ls}, {\tt
				ds}, {\tt Ds} may be followed by an integer
				expression $k_i$ in parentheses specifying the
				number of variables $o_i$ refers to (\eg {\tt
				(lp(3), dp(2))}).\cr
\subsec{Extra weight vector}
a($w_1$, $\ldots$)&		any of the above degree orderings may be
				preceded by an extra weight vector\cr
\noalign{\eject}
\subsec{Module orderings}
({\it c}, $o_1$, $\ldots$)&	sort by components first\cr
($o_1$, $\ldots$, {\it c\/})&	sort by variables first\cr
&				$o_i$ may be any of the above orderings or an
				extra weight vector, {\it c\/} may be one of
				{\tt C} or {\tt c}:\cr
C&				sort generators in ascending order (\ie {\tt
				gen({\it i\/})} $<$ {\tt gen({\it j\/})} iff $i <
				j$)\cr
c&				sort generators in descending order\cr
\endsec

\sec Data types(1.5cm)
\sectext
Examples of ring-independent types:\cr
\entryskip
\longitem
int i1 = 101; int i2 = 13 div 3;\cr
\entryskip
\longitem
intvec iv = 13 div 3, -4, i1;\cr
\entryskip
\longentry
intmat im[2][2] = 13 div 3, -4, i1;&
				a $2\times 2$ matrix.  Entries are filled
				row-wise, missing entries are set to zero, extra
				entries are ignored.  vector/matrix elements are
				accessed using the {\tt[$\ldots$]} operator,
				where the first element has index one (\eg {\tt
				iv[3]; im[1, 2];}).\cr
\entryskip
\longitem
string s1 = "a quote \char"5C " and a backslash \char"5C \char"5C";\par
string s2 = "con" + "catenation";\cr
\entryskip
\sectext
Basering in the following is {\tt ring r = 0, (x, y, z, mu, nu), dp;}\cr
\entryskip
\longitem
number n = 5/3;\cr
\entryskip
\longentry
poly p(1) = 3/4x3yz4+2xy2;\par
poly p(2) = (5/3)*mu\^{}2*nu\^{}3+n*yz2;&
				{\tt p(1)} equals $3/4x^3yz^4+2xy^2$.  Short
				format of mono\-mials is valid for one-character
				ring variables only.\cr
\entryskip
\longentry
ideal i = p(1..2), x+y;&	note the use of indexed names\cr
\entryskip
\longentry
vector v = [p(1), p(2), x+y];\par
vector w = 2*p(1)*gen(6)+n*nu*gen(1);&
				vectors may be written in brackets ({\tt
				[$\ldots$]}) or expressed as linear
				combinations of the canonical generators {\tt
				gen({\it i\/})}\cr
\entryskip
\longitem
module mo = v, w, x+y*gen(1);\cr
\entryskip
\longitem
resolution r = sres(std(mo), 0);\cr
\entryskip
\longentry
matrix ma[2][2] = 5/3, p(1), 101;&
				the rules for declaring, filling, and accessing
				integer matrices apply to types {\tt matrix}
				and {\tt vector}, too\cr
\entryskip
\longentry
list l = iv, v, p(1..2), mo;&
				lists may collect objects of any type.  They are
				ring-dependent iff one of the entries is.\cr
\entryskip
\longentry
def d = read("MPfile:r example.mp");&
				a name of type {\tt def} inherits the type of
				the object assigned first to it.  Useful if the
				actual type of an object is unknown.\cr
\endsec

\sec Monitoring and debugging tools()
timer = 1;&			print time used for commands to execute\cr
\longentry int t = timer; {\it command\/}; \rep; timer-t;&
				print time used for {\it commands\/} to execute\cr
memory(1);&			print number of bytes allocated from system\cr
option(prot);&			show algorithm protocol\cr
option(mem);&			show algorithm memory usage\cr
\entryskip
TRACE = 1;&			print protocol on execution of procedures\cr
listvar(all);&			list all (user-)defined names\cr
\longentry listvar({\it ringname\/});&
				list all names belonging to {\it ringname}\cr
\endsec

\eject

\sec Options()
option();&			show current option settings\cr
\longentry option($option_1$, no$option_2$, $\ldots$);&
				switch $option_1$ on and $option_2$ off, resp.\cr
option(none);&			reset all options to default values\cr
\sectext
Type {\tt help option;} for a list of all options.\cr
\subsec{Monitoring}
debugLib&			show loading of procedures from libraries\cr
mem&				show algorithm memory usage\cr
prot&				show algorithm protocol\cr
\subsec{Standard bases}
fastHC&				try to find highest corner as fast as possible\cr
intStrategy&			avoid divisions\cr
morePairs&			create additional pairs\cr
notSugar&			disable sugar strategy\cr
redSB&				compute reduced standard bases\cr
redTail&			reduce tails\cr
sugarCrit&			use sugar criteria\cr
weightM&			automatically compute weights\cr
\subsec{Resolutions}
minRes&				do additional minimizing\cr
notRegularity&			disable regularity bound\cr
\subsec{Miscellany}
returnSB&			let some functions return standard bases\cr
\endsec

\sec System variables()
\sectext
Type {\tt help System variables;} for a list of all system variables.\cr
\subsec{Standard bases}
degBound&			stop if (weighted) total degree exceeds {\tt
				degBound}\cr
multBound&			stop if multiplicity gets smaller than {\tt
				multBound}\cr
noether&			cut off all monomials above monomial {\tt
				noether}\cr
\subsec{Miscellany}
basering&			current basering\cr
minpoly&			minimal polynomial for algebraic extensions\cr
short&				do not print monomials in short format if zero\cr
timer&				on assignment of a non-zero value show time
				used for execution of executed commands.  On
				evaluation, return system time in seconds used
				by \Singular\ since start\cr
TRACE&				print information on procedures being executed
				if larger than one\cr
\endsec

\sec Input and output()
< "{\it filename\/}";&		load and execute {\it filename\/}\cr
\longentry write("{\it filename\/}", {\it expression}, \rep)&
				write {\it expressions\/} to ASCII file {\it
				filename}\cr
\longentry read("{\it filename\/}");&
				read ASCII file {\it filename\/} and return
				content as a string.  See also example below.\cr
\longentry
dump("MPfile: {\it filename\/}");\par
getdump("MPfile: {\it filename\/}");&
				dump current state of {\sc Singular} to {\it
				filename} and retrieve it, resp.\cr
\entryskip
\sectext
An example how to write one single expression (in this case the
ideal {\tt i}) to a file and read it back from there:

{\tt
write("i.save", i);\par
execute("ideal i=" + read("i.save") + ";");
}\cr
\endsec

\eject
				
\sec Libraries()
LIB "{\it library\/}";&		load {\it library}\cr
help {\it library\/};&		show help on {\it library}\cr
help all.lib;&			show list of all libraries\cr
\endsec

\sec Mapping(1.5cm)
\longentry map {\it name\/} = {\it ringname}, {\it ideal\/};&
				declare a map {\it name\/} from {\it ringname\/}
				to current basering.  The $i$-th ring variable
				from {\it ringname\/} is mapped to the $i$-th
				generator of {\it ideal}.\cr
\longentry {\it mapname\/}({\it expression\/})&
				apply map {\it mapname\/} to {\it expression}\cr
\entryskip
\sectext
Coefficients between rings with different basefields are mapped in the following
way (non-canonical maps only):\strut
\abovedisplayskip=0pt
\belowdisplayskip=0pt
$$
\eqalign{Z_p \rightarrow Q&  :[i]_p \mapsto i \in [-p/2,p/2] \subset Z\cr
	 Z_p \rightarrow Z_q&:[i]_p \mapsto i \in [-p/2,p/2] \subset Z, i \mapsto [i]_q}
$$
\cr
\noalign{\vskip -9pt} % dirty trick, gobbles the trailing \strut from \sectext
\entryskip
\longentry fetch({\it ringname}, {\it name\/})&
				map from ring {\it ringname\/} to current
				base\-ring.  The rings have to be identical up
				to names of ring variables\cr
\longentry imap({\it ringname}, {\it name\/})&
				map from subring {\it ringname\/} to current
				basering\cr
\longentry subst({\it expression}, {\it ringvar}, {\it monomial\/})&
				substitute {\it ringvar\/} by {\it monomial\/}
				in {\it expression}\cr
\endsec

\sec Miscellany(1.5cm)
\longentry setring({\it ringname\/})&
				make {\it ringname\/} the current basering\cr
\subsec{Data on polynomials}
\longentry ord({\it poly\/\alt vector\/})&
				return (weighted) degree of initial term\cr
\longentry deg({\it poly\/\alt vector\/})&
				return maximal (weighted) degree\cr
\longentry
size({\it ideal\/\alt module\/})\par
size({\it poly\/\alt vector\/})\par
size({\it string\/\alt intvec\/\alt list\/})&
				return (1) number of non-zero generators;
				(2) number of monomials; (3) length\cr
\longentry lead({\it expression\/})&
				return initial term(s)\cr
\subsec{Operations on polynomials}
\longentry gcd({\it $\hbox{poly}_1$}, {\it $\hbox{poly}_2$\/})&
				return greatest common divisor\cr
\longentry factorize({\it poly\/}\opt{, {\it int\/}})&
				return irreducible factors.  Return constant
				factor and multiplicities in dependency on {\it
				int}.\cr
\endsec

\sec Differentiation and jets(1.5cm)
\longentry
diff({\it expression}, {\it ringvar\/})\par
diff({\it $\hbox{ideal}_1$}, {\it $\hbox{ideal}_2$\/})&
				(1) return partial derivation by {\it
				ringvar\/}; (2) differentiate each elt.\ of {\it
				$\hbox{ideal}_2$\/} by the differential
				operators corres\-pon\-ding to the elements of {\it
				$\hbox{ideal}_1$}\cr
\longentry jacob({\it poly\/\alt ideal\/})&
				return jacobi ideal or matrix, resp.\cr
\longentry jet({\it expression}, {\it int\/}\opt{, {\it intvec\/}})&
				return {\it int\/}-jet of {\it expression}.
				Return weighted {\it int\/}-jet if {\it
				intvec\/} is specifified.\cr
\endsec

\eject

\sec Standard bases(1.5cm)
\longentry groebner({\it ideal\/\alt module\/}\opt{, {\it int\/}})&
				compute a standard basis (SB) of {\it ideal\/}
				resp.\ {\it module\/} using a heuristically
				chosen method.  Delimit com\-pu\-tation time to
				{\it int\/} seconds.\cr
\longentry std({\it ideal\/\alt module\/}\opt{, {\it intvec\/}})&
				compute a SB.  Use first Hilbert series {\it
				intvec\/} (result from {\tt hilb($\ldots$, 1)})
				for Hilbert-driven computation.\cr
\longentry stdfglm({\it ideal\/}\opt{, {\it string\/}})&
				use FGLM algorithm to compute a SB from a SB
				w.r.t.\ the ``simpler'' ordering {\it string\/}
				(de\-faults to {\tt dp})\cr
\longentry stdhilb({\it ideal\/}\opt{, {\it intvec\/}})&
				use Hilbert-driven algorithm to compute a SB.
				If Hil\-bert series {\it intvec\/} is not
				specified compute it first.\cr
\longentry fglm({\it ringname}, {\it idealname\/})&
				use FGLM algorithm to transform SB {\it
				idealname\/} from ring {\it ringname\/} to a SB
				w.r.t.\ the ordering of the current basering\cr
\longentry reduce({\it expression}, {\it ideal\/\alt module\/}\opt{, {\it int\/}})&
				reduce {\it expression\/} w.r.t.\ second
				argument which should be a SB.  Use lazy
				reduction if {\it int\/} equals one.\cr
\endsec

\sec Computation of invariants(1.5cm)
\sectext
Most of the results are meaningful only if the input ideal or module is
represented by a standard basis.\cr
\longentry degree({\it ideal\/\alt module\/})&
				display (Krull) dimension, codimension and
				multiplicity\cr
\longentry dim({\it ideal\/\alt module\/})&
				return (Krull) dimension\cr
\longentry hilb({\it ideal\/\alt module\/}\opt{, {\it int\/}})&
				display first and second Hilbert series with one
				argument.  Return {\it int}-th Hilber series
				otherwise (${\it \hbox{int}} = 1,2$).\cr
\longentry mult({\it ideal\/\alt module\/})&
				return multiplicity\cr
\longentry vdim({\it ideal\/\alt module\/})&
				return vector space dimension of current
				basering modulo {\it ideal\/} or {\it module},
				resp.\cr
\endsec

\sec Resolutions(1.5cm)
\sectext
An integer argument {\it length\/} in the following descriptions specifies the
length of the resolution to compute.  If {\it length\/} equals zero, the whole
resolution is computed.\cr
\longentry res({\it ideal\/\alt module}, {\it length\/}\opt{, {\it int\/}})&
				compute a free resolution (FR) of {\it ideal\/}
				resp.\ {\it module\/} using a heuristically
				chosen method.  Compute a minimal resolution if
				a third argument is given.\cr
\longentry mres({\it ideal\/\alt module}, {\it length\/})&
				compute a minimal FR using the standard basis
				method\cr
\longentry lres({\it ideal\/\alt module}, {\it length\/})&
				compute a FR using LaSacala's method\cr
\longentry sres({\it ideal\/\alt module}, {\it length\/})&
				compute a FR using Schreyer's method\cr
\longentry syz({\it ideal\/\alt module\/})&
				compute the first syzygy\cr
\longentry minres({\it resolution\/\alt list\/})&
				minimize a free resolution\cr
\longentry betty({\it resolution\/\alt list\/})&
				compute the graded Betti numbers of a module
				represented by a resolution\cr
\endsec

\bye
